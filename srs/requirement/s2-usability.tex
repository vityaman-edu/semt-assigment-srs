% [This section should include all of those requirements
%  that affect usability. For example,
% - specify the required training time for a normal
%   users and a power user to become productive at
%   particular operations
% - specify measurable task times for typical tasks
%   or base the new system’s usability requirements
%   on other systems that the users know and like
% - specify requirement to conform to common usability
%   standards, such as IBM’s CUA standards Microsoft’s 
%   GUI standards]

\begin{enumerate}
  \item Сайт должен быть разделен на 2 раздела:
        \begin{enumerate}
          \item Для путешествий
          \item Для грузовых перевозок
        \end{enumerate}

  \item Титульная страница раздела для путешествий
        должна содержать ленту последних новостей,
        онлайн-табло с рейсами, поиск рейсов по фильтрам,
        раздел "Популярное", содержащий ссылки на
        страницы с описанием предоставляемых услуг и
        раздел "Как добраться", показывающий все
        возможные варианты пути до аэропорта.

  \item Все страницы кроме титульной должны подчиняться
        установленному далее макету страницы: \\
        Макет состоит из 4 частей:

        \begin{enumerate}
          \item Шапка, содержащая телефон горячей линии
                аэропорта, возврат на главную страницу и все
                доступные языки, на которых возможно
                просматривать сайт. Также в шапке располагается
                интсрумент для переключения между разделами
                сайта.

          \item Часть со всеми разделами информационных
                страниц, ссылками на социальные сети аэропорта
                Домодедово и поисковой строкой.

          \item Главная часть, отображающая информацию
                выбранного раздела, а также поиск рейсов и разделы
                "Популярное" и "Как добраться".

          \item Футер, содержащий ссылкы на форму обратной
                связи, разделы партнёрских возможностей и
                юридическую документацию.
        \end{enumerate}

\end{enumerate}
