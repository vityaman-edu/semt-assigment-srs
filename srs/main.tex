\documentclass{article}

\usepackage[utf8]{inputenc}
\usepackage[russian]{babel}
\usepackage[a4paper, margin=1in]{geometry}
\usepackage{graphicx}
\usepackage{amsmath}
\usepackage{wrapfig}
\usepackage{multirow}
\usepackage{mathtools}
\usepackage{pgfplots}
\usepackage{pgfplotstable}
\usepackage{setspace}
\usepackage{changepage}
\usepackage{caption}
\usepackage{csquotes}
\usepackage{hyperref}
\usepackage{listings}

\pgfplotsset{compat=1.18}
\hypersetup{
  colorlinks = true,
  linkcolor  = blue,
  filecolor  = magenta,      
  urlcolor   = darkgray,
  pdftitle   = {
    semt-report-1-srs-smirnov-shinakov
  },
}

\begin{document}

\begin{titlepage}
  \begin{center}
    \begin{spacing}{1.4}
      \large{Университет ИТМО} \\
      \large{Факультет программной инженерии и компьютерной техники} \\
    \end{spacing}
    \vfill
    \textbf{
      \huge{Методы и средства программной инженерии.} \\
      \huge{Лабораторная работа №1.} \\
      \huge{Software Requirements Specification} \\
    }
  \end{center}
  \vfill
  \begin{center}
    \begin{tabular}{r l}
      Смирнов Виктор Игоревич  & P32131 \\
      Шиняков Артём Дмитриевич & R32372 \\
      Вариант                  & 1025   \\
    \end{tabular}
  \end{center}
  \vfill
  \begin{center}
    \begin{large}
      2023
    \end{large}
  \end{center}
\end{titlepage}

\tableofcontents

\section{Введение}

\subsection{Цель}
[Specify the purpose of this SRS. The SRS should 
fully describe the external behavior of the 
application or subsystem identified. It also 
describes nonfunctional requirements, design 
constraints and other factors necessary to 
provide a complete and comprehensive description 
of the requirements for the software.]

Цель - создание сайта для международного
аэропорта Домодедово, расположенного в 
городе Москва. Требования к сайту описаны ниже.


\subsection{Краткая сводка возможностей}

Сайт состоит из двух разделов для путешествий и бизнеса.

\subsubsection{Раздел для путешествий}

Раздел сайта, предоставлет:
\begin{enumerate}
      \item информацию о состоянии аэропорта и рейсов,
            у которых одним из пунктов назначения/вылета
            указан Домодедово,
      \item ленту новостей, касающиеся данного аэроузла,
      \item возможность бронирования билетов на перелеты,
      \item план местности для ориентации пассажиров,
            подробные указания пути до Домодедово, сведения
            о доступных поблизости местах отдыха/ночлега
      \item юридические документы, содержащие описание
            прав и обязанностей пассажиров и посетителей
      \item способы связи с представителями данной
            организации
\end{enumerate}

Взаимодействие с клиентом осуществляется 2 способами:
\begin{enumerate}
      \item Клиент связывается с компанией через контакты
            или социальные, указанные на сайте
      \item Клиент бронирует места на рейс или в гостинице,
            указывая свои персональные данные
\end{enumerate}

Сервис должен поддерживать роботоспособность сайта,
транзакции при оплате билетов и предоставлять актуальную
информацию о состоянии полетов.

\subsubsection{Раздел для бизнеса}

Этот раздел сайта, предоставляет информацию о :
\begin{enumerate}
      \item состоянии аэропорта и рейсов, расписании
      \item грузовых рейсах и перевозимых грузах
      \item возмоностях для сотрудничества
      \item инфраструктуре
      \item карьерных возможностях
      \item о контактах для обратной связи и ссылки
            на социальные сети аэропорта Домодедово
\end{enumerate}


\subsection{Определения, акронимы и сокращения}
% [This subsection should provide the definitions 
% of all terms, acronyms, and abbreviations required 
% to properly interpret the SRS.  This information 
% may be provided by reference to the project 
% Glossary.]

% TODO: Заполняем по ходу или уже в готовом варианте

\subsection{Ссылки}

\begin{itemize}
    \item Редактор Use-case диаграмм:
          \url{https://creately.com/diagram-type/use-case/}
    \item SRS шаблон:
          \url{https://docs.google.com/document/d/11aTUhjJxHqDMJGTDXDKh_8U_f2YVWKBS/edit?usp=sharing&ouid=112239579841283566048&rtpof=true&sd=true}
\end{itemize}


\subsection{Обзор}
[This subsection should describe what the rest 
of the SRS contains and explain how the document 
is organized.]


\section{Общее описание}

\subsection{Функционал продукта}
Cайт должен давать возможность отслеживания 
всех грузовых и пассажирских рейсов вылетающих из/прибывающих в 
аэропорт Домодедово, поиска полетов по их 
параметрам и атрибутам, прочтения новостей, 
связанных с аэропортом, отслеживания состояния 
аэропорта, услуг и возможностей, которые он предоставляет. 
Также разрабатываемое решение должно предоставлять 
услугу бронирования и приобретения авиабилетов, 
отображения избранных рейсов клиента, отображение плана территории, 
возможность бронирования номеров в отелях и гостиницах города Москва, 
покупка парковочного места для авторизованных пользователей,
возможные маршруты до аэропорта и обратную 
связь с представителями Домодедово. 

\subsection{Описание пользователей}
\begin{enumerate}
    \item Клиенты, обладающие необходимостью в приобретении билета или отслеживании состояния 
    определенного рейса
    \item Пассажиры, желающие забронировать место проживания или парковочное пространство 
    по прибытии или заранее
    \item Администраторы и работники аэропорта Домодедово
    \item Представители компаний авиаперевозок, пользующиеся пространством аэропорта
\end{enumerate}


\subsection{Факторы и зависимости}
% TODO: here we should describe what we need 
% to launch our project

\begin{enumerate}
    \item Для реализации оплаты услуг онлайн прямо
          на нашем сайте, необходима интеграция с 
          популярными платёжными системами: 
          MirPay, YandexPay, YoMoney.
          
    \item Для отображения актуального
          расписания и состояния полетов необходимо 
          соединение с системами авиакомпаний, 
          обслуживаемых аэропортом Домодедово.
\end{enumerate}



\subsection{Ограничения}
На разрабатываемый сайт накладываются ограничения изложенные в этом файле.
Незадокументированные требования не учитываются при создании технического решения задач.
Ограничениями на минимальный функционал сайта являются требовнаия из раздела функциональных.
Ограниениями на используемые технологии являются требования из раздела нефункциональных.

TODO: после нефункциональных посмотрим, мб допишем.

\section{Технические требования}

\subsection{Функциональность}
[This section describes the functional requirements 
of the system for those requirements which are 
expressed in the natural language style. For many
applications, this may constitute the bulk of 
the SRS Package and thought should be given to 
the organization of this section. This section 
is typically organized by feature, but alternative
organization methods may also be appropriate, 
for example, organization by user or organization 
by subsystem.  Functional requirements may include 
feature sets, capabilities, and security.

Where application development tools, such as requirements 
tools, modeling tools, etc., are employed to capture the 
functionality, this section document will refer to the 
availability of that data, indicating the location and 
name of the tool that is used to capture the data.]



\begin{enumerate}
    \item Пользователь должен видеть оповещения о срочных/временных изменениях работы аэропорта
    Домодедово на главной странице сайта.

    \item Пользователь должен иметь возможность прочтения новостей, связанных с аэропортом
    Домодедово.

    \item Пользователь должен иметь возможность заполнения и отправки формы обратной связи.
    
    \item Страница с новостями должна предусматривать функцию фильтрации по дате публикаций
    и по теме новостных записей.

    \item Сайт должен содержать контакты для связи с представителями аэропорта Домодедово и ссылки на социальные контакты.
    
    \item Администраторы должны иметь доступ к базе данных клиентов сервиса PARKING.DME.RU.
    
    \item Администраторы должны иметь возможность добавления новых новстей в ленту.
    
    \item Администраторы должны иметь возможность создавать и удалять экстренные/временные оповещения об изменениях в работе аэропорта.
    
    \item Администраторы должны иметь возможность отслеживания заполненных форм обратной связи.
    
    \item Разрабатываемый раздел сайта должен позволять пользователю просматривать табло с расписанием.
    полетов, начальным или конечным пунктом которых указан аэропорт Домодедово.

    \item Онлайн табло, отображающее расписание рейсов, должно иметь функцию
    поиска по номеру рейса/направлению полета.

    \item Пользователь должен иметь возможность поиска и отслеживания рейсов по пункту назначения, 
    авиакомпании, предоставляющей услуги, или направлению, и дате отправления.

    \item Пользователь должен иметь возможность бронирования и покупки авиабилетов
    на рейсы, имеющие свободные места.

    \item Пользователь должен иметь возможность регистрации личного кабинета партнера или пассажира. Для пассажиров долен быть огранизован
    доступ к бронировнаию парковочных мест на сайте PARKING.DME.RU. Для партнеров доступны модули для работы с cargo.
    
    \item Пользователь должен иметь возможность отслеживания груза по его накладному номеру.
    
    \item Пользователь должен иметь возможность заполнения и отправки анкеты соискателя.
    
    \item Администратор должен иметь возможность отслеживания заполненных анкет соискателя.
    
    \item Пользователь должен иметь возможность просмотра юридических документов, определяющих условия работы аэропорта и пердоставления услуг 
    авиакомпаниями.

    \item Пользователь должен иметь возможность просмотра информационных страниц с описанием услуг и условий пользования аэропортом.
    Также должна предоставляться возможность печати отдельных экземпляров правил.
\end{enumerate}

\subsection{Требования}

\begin{enumerate}
    \item Разрабатываемый раздел сайта должен позволять пользователю
          просматривать табло с расписанием полетов, начальным или
          конечным пунктом которых указан аэропорт Домодедово. \\
          Трудоёмкость: 8. Риск: Средний. Приоритет: Высокий.

    \item Онлайн табло, отображающее расписание рейсов должно иметь
          функцию поиска по номеру рейса/направлению полета. \\
          Трудоёмкость: 8. Риск: Средний. Приоритет: Высокий. \\
          Риск: потеря связи с сервисами, предоставляющими расписание рейсов.

    \item Онлайн табло, отображающее расписание рейсов должно иметь
          функцию поиска по номеру рейса/направлению полета. \\
          Трудоёмкость: 8. Риск: Средний. Приоритет: Высокий.

    \item Страница с новостями должна предусматривать функцию
          фильтрации по дате публикаций и по теме новостных записей. \\
          Трудоёмкость: 8. Риск: Низкий. Приоритет: Средний.
          
    \item Сайт должен содержать контакты для связи с представителями
          аэропорта Домодедово и ссылки на социальные контакты \\
          Трудоёмкость: 2. Риск: Низкий. Приоритет: Высокий. \\
          Риск: замена ссылок на вредоносные в случае атак на безопасность

    \item Администраторы должны иметь доступ к базе данных клиентов
          сервиса PARKING.DME.RU. \\
          Трудоёмкость: 6. Риск: Средний. Приоритет: Высокий. \\
          Риск: Потеря связи с сервисом, Потеря данных пользователей во 
          время запроса

    \item Администраторы должны иметь возможность добавления новых
          новстей в ленту. \\
          Трудоёмкость: 4. Риск: Низкий. Приоритет: Средний.

    \item Администраторы должны иметь возможность создавать и удалять
          экстренные/временные оповещения об изменениях в работе аэропорта. \\
          Трудоёмкость: 4. Риск: Средний. Приоритет: ОчВысокий. \\
          Риск: добавление вредоностной ссылки под видом ссылки на новость

    \item Администраторы должны иметь возможность отслеживания
          заполненных форм обратной связи. \\
          Трудоёмкость: 2. Риск: Низкий. Приоритет: Низкий. \\
          Риски: потеря актуальности оповещения

    \item Администратор должен иметь возможность отслеживания
          заполненных анкет соискателя. \\
          Трудоёмкость: 2. Риск: Низкий. Приоритет: Средний.

    \item Пользователь должен видеть оповещения о срочных/временных
          изменениях работы аэропорта Домодедово на главной странице сайта. \\
          Трудоёмкость: 6. Риск: Средний. Приоритет: ОчВысокий. \\
          Риски: потеря актуальности оповещений.
          
    \item Пользователь должен иметь возможность прочтения новостей,
          связанных с аэропортом Домодедово. \\
          Трудоёмкость: 8. Риск: Низкий. Приоритет: Средний.

    \item Пользователь должен иметь возможность прочтения новостей,
          связанных с аэропортом Домодедово. \\
          Трудоёмкость: 8. Риск: Низкий. Приоритет: Средний. \\
          Риски: переход по вредоностной ссылке в случае атаки

    \item Пользователь должен иметь возможность заполнения и 
          отправки формы обратной связи. \\
          Трудоёмкость: 4. Риск: Низкий. Приоритет: Средний.

    \item Пользователь должен иметь возможность поиска и отслеживания 
          рейсов по пункту назначения, авиакомпании, предоставляющей услуги, 
          или направлению, и дате отправления. \\
          Трудоёмкость: 8. Риск: Средний. Приоритет: Высокий. \\
          Риски: потеря актуальности данных, что приведет к 
          ошибочному информированию

    \item Пользователь должен иметь возможность бронирования и 
          покупки авиабилетов на рейсы, имеющие свободные места. \\
          Трудоёмкость: 16. Риск: Средний. Приоритет: Высокий. \\
          Риски: \\
          1. Потеря данных во время транзакции \\
          2. Потеря связи между транзакцией и получением билета.\\
          3. Утечка данных о пользователе

    \item Пользователь должен иметь возможность регистрации личного 
          кабинета партнера или пассажира. Для пассажиров долен быть 
          огранизован доступ к бронировнаию парковочных мест на сайте 
          PARKING.DME.RU. Для партнеров доступны модули для работы с cargo. \\
          Трудоёмкость: 8. Риск: Низкий. Приоритет: Высокий.
<<<<<<< HEAD
          Риски: \\
          1. Потеря данных во время транзакции \\
          2. Потеря связи между транзакцией и бронированием места. \\
          3. Утечка данных о пользователе
          
    \item Пользователь должен иметь возможность отслеживания груза 
          по его накладному номеру. \\
          Трудоёмкость: 6. Риск: Средний. Приоритет: Высокий. \\
          Риски: потеря актуальности данных.

    \item Пользователь должен иметь возможность заполнения и отправки 
          анкеты соискателя. \\
          Трудоёмкость: 4. Риск: Низкий. Приоритет: Средний.

=======
    \item Пользователь должен иметь возможность отслеживания груза 
          по его накладному номеру. \\
          Трудоёмкость: 6. Риск: Средний. Приоритет: Высокий.
    \item Пользователь должен иметь возможность заполнения и отправки 
          анкеты соискателя. \\
          Трудоёмкость: 4. Риск: Низкий. Приоритет: Средний.
>>>>>>> 24d340f (resolved conflicts)
    \item Пользователь должен иметь возможность просмотра юридических 
          документов, определяющих условия работы аэропорта и пердоставления 
          услуг авиакомпаниями. \\
          Трудоёмкость: 2. Риск: Низкий. Приоритет: Низкий.
<<<<<<< HEAD

=======
>>>>>>> 24d340f (resolved conflicts)
    \item Пользователь должен иметь возможность просмотра информационных 
          страниц с описанием услуг и условий пользования аэропортом. 
          Также должна предоставляться возможность печати отдельных страниц. \\
          Трудоёмкость: 8. Риск: Средний. Приоритет: Высокий.
<<<<<<< HEAD

=======
>>>>>>> 24d340f (resolved conflicts)
\end{enumerate}


\subsection{Риски}

\begin{enumerate}
      \item \textbf{Потеря связи с сервисами, предоставляющими расписание рейсов.} \\
            Сервисы, предсотавляющие расписание рейсов могут отказать
            в обслуживании нашим сервисам из-за внутреннего сбоя,
            либо мы потеряем связь с ними из-за сетевых проблем \\
            Вероятность: Низкая. \\
            Опасность: Средняя. \\
            Смягчение: Отображать на табло последнее состояние расписания, но с
            отметкой, что данные не актуальны и были загружены тогда-то.

      \item \textbf{Замена злоумышлиниками ссылок и контактов в случае взлома} \\
            Злоумышлиники в результате успешной хакерской атаки получат привелигерованный
            доступ к системе и смогут разместить вредоносные ссылки и подменить контакты
            на сайте, что поставит наших клиентов под угрозу. \\
            Вероятность: Низкая. \\
            Опасность: Высокая. \\
            Последствия: Утечка персональных данных, потеря доверия пользователей. \\
            Смягчение: Вставлять контактную информацию в верстку прямо на этапе
            сборки проекта (захардкодить), оповестить пользователей о взломе,
            сделать веб-сайт недоступным, добавить проверку того, что ссылки содержат 
            наш домен.

      \item \textbf{Размещение фейковой информации в новостях сотрудником-злоумышлиником} \\
            Злоумышленник будучи администратором веб-сайта сможет разместить фейковую новость,
            наши клиенты будут в опасности.
            Вероятность: Средняя. \\
            Опасность: Высокая. \\
            Последствия: Хакерские атаки на посетителей сайта, потеря репутации. \\
            Смягчение: Ввести контроль - обязательное ревью перед публикацией записи. \\

      \item \textbf{Отказ ДЦ} \\
            Датацентр, на котором развернуто наше приложение может отказать, например,
            из-за отключения электроэнергии или ошибки персонала. Из-за этого наш облачный
            сервис будет недоступен. \\
            Вероятность: Средняя. \\
            Опасность: Очень-очень высокая. \\
            Последствия: Сервис не доступен. \\
            Смягчение: поднимать сразу несколько stateless сервисов,
            балансировать нагрузку, реплецировать базу данных и кеши,
            разворачивать инфраструктуру как минимум в 2 разных ДЦ.

      \item \textbf{Отказ платежной системы} \\
            Одна из платежных систем может отказать в обслуживании из-за 
            внутренних проблем. \\
            Вероятность: Крайне низкая. \\
            Опасность: Высокая. \\
            Последствия: Невозможность купить билеты и забронировать 
            парковочное место онлайн на нашем сайте. \\
            Смягчение: Предупреждать пользователей о недоступности платежа в 
            данный момент, предложить воспользоваться другой ПС.
      
      \item \textbf{Спам в анкетах соискателя} \\
            Злоумышлиники могут спамить анкетами соискателя, тем самым
            не давая администраторам разбирать поступающие от обычных 
            пользователей сообщения. \\
            Вероятность: Высокая. \\
            Опасность: Низкая. \\
            Последствия: Снижение эффективности работы администраторов. \\
            Смягчение: Добавить антиспам защиту.

      \item \textbf{Перебор номеров карт пользователей и прочее мошенничество} \\
            Злоумышлиники могут перебирать карты пользователей с целью покупуки
            билетов за их счет. \\
            Вероятность: Высокая. \\
            Опасность: Высокая. \\
            Последствия: Кража денег со счетов пользователей. \\
            Смягчение: Добавление анти-фрод функциональности и способа возврата средств,
            добавление антиробота. 
      
      \item \textbf{Взлом аккаунтов клиентов} \\
            Злоумышленники могут взломать аккаунты клиентов. \\
            Вероятность: Высокая. \\
            Опасность: Высокая. \\
            Последствия: Утечка персональных данных, хакерские атаки на пользователей. \\
            Смягчение: Просить пользователей выбирать сложные пароли, требовать 
            стойкие пароли при регистрации.

      \item \textbf{Сбой сервиса из-за ошибки программиста} \\
            Программист может случайно положить какой-нибудь сервис.
            Вероятность: Очень высокая. \\
            Опасность: Очень высокая. \\
            Последствия: Неопределенное поведение системы. \\
            Смягчение: Строгие требования к разработке: 
            юнит-тесты, код-ревью, тестовая среда, выкатка в прод с апрувом,
            выкатка в прод в определенное время, настроенные алерты и прочие 
            инструменты наблюдаемости, возможность быстрого отката,
            возможность канареечных релизов.

      \item \textbf{DDoS атака} \\
            Злоумышленники могут устроить DDoS атаку на нашу инфраструктуру и 
            вызвать отказ в обслуживании.
            Вероятность: Средняя. \\
            Опасность: Высокая. \\
            Последствия: Недоступность сервиса в течении атаки. \\
            Смягчение: Добавить балансировщик нагрузки, антиробот, 
            обеспечить наилучшую производительность системы.

\end{enumerate}

\subsection{Удобство использования}

\begin{enumerate}
      \item Сайт должен быть разделен на 2 раздела:
            \begin{enumerate}
                  \item Для путешествий
                  \item Для грузовых перевозок
            \end{enumerate}
            Приоритет: Высокий. Стабильность: Высокая. Трудоемкость: 10 sp
            Риски: 1, 2, 11

      \item Титульная страница раздела для путешествий
            должна содержать ленту последних новостей,
            онлайн-табло с рейсами, поиск рейсов по фильтрам,
            раздел "Популярное", содержащий ссылки на
            страницы с описанием предоставляемых услуг и
            раздел "Как добраться", показывающий все
            возможные варианты пути до аэропорта.
            
            Приоритет: Высокий. Стабильность: Средняя. Трудоемкость: 12 sp
            Риски: 1, 2, 11


      \item Все страницы кроме титульной должны подчиняться
            установленному далее макету страницы: \\
            Макет состоит из 4 частей:

            \begin{enumerate}
                  \item Шапка, содержащая телефон горячей линии
                        аэропорта, возврат на главную страницу и все
                        доступные языки, на которых возможно
                        просматривать сайт. Также в шапке располагается
                        интсрумент для переключения между разделами
                        сайта.

                  \item Часть со всеми разделами информационных
                        страниц, ссылками на социальные сети аэропорта
                        Домодедово и поисковой строкой.

                  \item Главная часть, отображающая информацию
                        выбранного раздела, а также поиск рейсов и разделы
                        "Популярное" и "Как добраться".

                  \item Футер, содержащий ссылкы на форму обратной
                        связи, разделы партнёрских возможностей и
                        юридическую документацию.
            \end{enumerate}

            Приоритет: Высокий. Стабильность: Высокая. Трудоемкость: 14 sp
            Риски: 1, 2, 11

\end{enumerate}


\subsection{Надежность}
[Requirements for reliability of the system should
 be specified here. Some suggestions follow:

- Availability—specify the percentage of time 
  available ( xx.xx\%), hours of use, maintenance 
  access, degraded mode operations, etc.

- Mean Time Between Failures (MTBF) — this is 
  usually specified in hours, but it could also 
  be specified in terms of days, months or years.

- Mean Time To Repair (MTTR)—how long is the 
  system allowed to be out of operation after it 
  has failed?

- Accuracy—specify precision (resolution) and 
  accuracy (by some known standard) that is required
  in the system’s output.

- Maximum Bugs or Defect Rate—usually expressed 
  in terms of bugs per thousand of lines of code 
  (bugs/KLOC) or bugs per function-point
  (bugs/function-point).
  
- Bugs or Defect Rate—categorized in terms of 
  minor, significant, and critical bugs: the 
  requirement(s) must define what is meant by 
  a “critical” bug; for example, complete loss 
  of data or a complete inability to use certain 
  parts of the system’s functionality.]


\subsection{Производительность}
% [The system’s performance characteristics should 
% be outlined in this section. Include specific 
% response times. Where applicable, reference 
% related Use Cases by name.
% 
% - response time for a transaction (average, maximum)
% 
% - throughput, for example, transactions per second
% 
% - capacity, for example, the number of customers or 
%   transactions the system can accommodate
% 
% - degradation modes (what is the acceptable mode 
%   of operation when the system has been degraded 
%   in some manner)
% 
% - resource utilization, such as memory, disk, 
%   communications, etc.


Сайт представляет из себя в основном информационные страницы и действия относительно
авиаперелетов, которым пользуются, как правило, заблаговременно. По этой причине разрабатываемое решение
не обязательно должно содержать максимальную оптимизацию относительно времени.

\begin{itemize}
    \item Среднее время транзакции: 2,8 с %показывает аналитика с сайта производительности
    \item Среднее время до первого байта: 1,6 с
\end{itemize}



\subsection{Поддерживаемость}

Каждый микросервис должен иметь следующую документацию:
\begin{enumerate}
    \item Описание доменной модели
    \item Описание таблиц в БД
    \item Open API спецификацию и Swagger страницу 
          всегда доступную в интранете
    \item Интро для нового сотрудника
    \item Базу данных всех тикетов за всю жизнь
    \item Документы с каждого митинга, в которых описаны 
          все принятые решения и указаны аргументы 
          за и против. Записи всех митингов, особенно 
          техтолков.
    \item Полная история работы системы контроля версий
          (например, чтобы найти автора того или иного класса).
    \item Главная страница сервиса со всеми ссылками
\end{enumerate}

Работники обсуждать в ЛС только личные вопросы, все рабочие 
не очень важные выносятся на обсуждение в рабочий чат,
а в основном общаться нужно только в комментариях к тикетам,
для сохранения истории.
\end{enumerate}

Работники обсуждать в ЛС только личные вопросы, все рабочие 
не очень важные выносятся на обсуждение в рабочий чат,
а в основном общаться нужно только в комментариях к тикетам,
для сохранения истории.

Работники обсуждать в ЛС только личные вопросы, все рабочие 
не очень важные выносятся на обсуждение в рабочий чат,
а в основном общаться нужно только в комментариях к тикетам,
для сохранения истории.

>>>>>>> 24d340f (resolved conflicts)
В каждом проекте должно быть настроено:
\begin{enumerate}
    \item При доступности технологии, автоматическая 
          генерирация API интерфейсов и моделей с помощью
          Open API Codegen, а также клиентов к ним.
    \item Линтеры и форматтеры, настроенные самым беспощадным способом.
    \item Автоматические тесты.
\end{enumerate}

Должна быть возможность выдачи админам прав к ограниченному 
просмотру содержимого базы данных на минимальный органиченный 
промежуток времени.

<<<<<<< HEAD
Во всех сервисах метрики собираются Прометеусом, 
отображаются в графане.

Должно быть настроено 2 среды: тестовая и продовая.
=======
Во всех сервисах метрики собираются Прометеусом, 
отображаются в графане.

>>>>>>> 24d340f (resolved conflicts)
Должно быть настроено 2 среды: тестовая и продовая.


\subsection{Интерфейсы}

\begin{enumerate}
      \item Пользовательский
            \begin{enumerate}
                  \item Сайт должен иметь удобное навигационное меню,
                        достаточно интуитивно-понятное, чтобы пользователь совершил не более
                        5 переходов.
                        В нем должны содержаться ссылки на главную страницу, страницу вылета или прилета,
                        страницу со всеми возмоными путями до аэропорта,
                        страницу паркинга, страницу с планом аэропорта.

                        Приоритет: Высокий. Стабильность: Высокая. Трудоемкость: 12 sp
                        Риски: 1, 2

                  \item Главная страница должны содержать основные интерактивные возможности:
                        онлайн-табло авиаполетов, фильтрацию рейсов по параметрам,
                        ссылки на популярные услуги сайта.

                        Приоритет: Высокий. Стабильность: Высокая. Трудоемкость: 8 sp
                        Риски: 1, 2

                  \item Информационные страницы должны отображать в основной секции только иформацию,
                        свуязанную с ее названием для простоты поиска нужной информации,
                        и возможность поиска рейсов, переходов на популярные услуги сайта.

                        Приоритет: Высокий. Стабильность: Высокая. Трудоемкость: 10 sp
                        Риски: 1, 2

                  \item Страница с планом аэропорта должна быть организована так, чтобы
                        посетитель мог легко ориентироваться по интерактивной карте и имел возможность
                        быстрого поиска нужной локации.

                        Приоритет: Высокий. Стабильность: Высокая. Трудоемкость: 10 sp
                        Риски: 1, 2

                  \item Страница с новостями должна содержать только новости и
                        поиск по названию для легкой навигации пользователя.

                        Приоритет: Высокий. Стабильность: Высокая. Трудоемкость: 10 sp
                        Риски: 1, 2

            \end{enumerate}
      \item Аппаратные
            Система должна быть защищена от хакерских атак и быть устойчива к высоким нагрузкам, благодаря
            VK Cloud Solution - облачному сервису, предоставляющему размещение серверов в своем пространстве.
            \\
            Приоритет: Высокий. Стабильность: Высокая. Трудоемкость: 12 sp
            Риски: 1, 2
            \\
      \item Программные
            \begin{itemize}
                  \item Сайт использует MirPay, YandexPay, YoMoney в качестве платежных систем для денежных транзакций.
                  \item Сайт использует Telegram, VK and Youtube APIs для ссылок на соц. сети аэропорта.
                  \item Сайт использует API сервисов компаний авиаперевозок для получения информации о рейсах.
            \end{itemize}
            \\
            Приоритет: Высокий. Стабильность: Высокая. Трудоемкость: 14 sp
            Риски: 1, 2

      \item Сетевые интерфейсы
            Сайт использует протокол IP для передачу данных через сеть интернет.
            \\
            Для связи клиента и сервера устанавливается HTTPS протокол.
            \\
            SSL и TLS используются для защищённой передачи данных.
            \\
            Приоритет: Высокий. Стабильность: Высокая. Трудоемкость: 10 sp
            Риски: 1, 2

\end{enumerate}

\subsection{Лицензия}
\input{3-requirement/s7-licence.tex}

\section{Прецеденты}
\subsection{Поиск рейса}
\textbf{Пользователь} ищет рейс по доступной информации.
В данном действии принимает участие \textbf{Посетитель}.
Однако для инициации данного события \textbf{Пользователь}
должен иметь данные, необходимые для поиска информации о
рейсе. \textbf{Пользователь} вводит данные в поля фильтра
рейса(таблицы). Сайт выводит рейсы по результатам запроса
Либо же \textbf{Сайт} выводит информацию об отсутствии
рейса с такими данными.

\subsection{Просмотр пользователем}
\textbf{Пользователь} просматривает тексты на информационных
страницах. Пользователь переходит на одну из информационных
страниц. Страница отображается пользователю.

\subsection{Заполнение формы}
\textbf{Пользователь} отправляет форму обратной связи или
анкеты соискателя. Участвуют в данном действии
\textbf{Посетитель} и \textbf{Администратор}.
Пользователь имеет повод для обращения к представительству
аэропорта. Пользователь вводит данные в форму и отправляет.
Данные формы доступны администратору. \textbf{Администратор}
принимает данные из формы. Форма не отправляется/не
доходит до администратора.

\subsection{Пользователь бронирует билеты}
Клиент покупает или бронирует билеты на рейс. Участники:
\textbf{Клиент}, \textbf{Публичный}, \textbf{Биллинг},
\textbf{Платежная система}, \textbf{Почтовый}. Пользователь
должен быть зарегестрирован. Действия:
\begin{enumerate}
      \item \textbf{Клиент} выбирает рейс.
      \item \textbf{Клиент} заполняет форму и отправляет ее
            в \textbf{Публичный}.
      \item \textbf{Публичный} бронирует билет, используюя
            сервис \textbf{Билетер}.
      \item \textbf{Публичный} просит у \textbf{Биллинга}
            форму для ввода реквизитов для оплаты.
      \item \textbf{Биллинг} регистрирует факт запросы формы
            для оплаты и просит \textbf{Платежную систему}
            дать форму, после чего пересылает ее
            \textbf{Публичному}, а тот \textbf{Клиенту}.
      \item \textbf{Клиент} вводит реквизиты карты и нажимает
            отправить, данные отправляются в.
            \textbf{Платежную систему}
      \item \textbf{Клиент} просит предоставить услугу у
            \textbf{Публичного}.
      \item \textbf{Публичный} справшивает у \textbf{Биллинга},
            как там с оплатой дела.
      \item \textbf{Биллинг} справшивает у \textbf{Платежной системы},
            как там с оплатой дела.
      \item Если все ОК, то
            \begin{enumerate}
                  \item \textbf{Биллинг} закрывает транзакцию, которую
                        открыл при регистрации запроса на платеж,
                        если фейл, то фейлит транзакцию, если ХЗ,
                        то отвечает ХЗ.
                  \item Если все ОК \textbf{Публичный} просит
                        \textbf{Билетера} выдать билет.
                  \item \textbf{Билетер} отмечает забронированный
                        билет как купленный и просит \textbf{Почтового},
                        отправить билет на почту \textbf{Пользователя}.
            \end{enumerate}
      \item Если ошибка с оплатой, то
            \begin{enumerate}
                  \item \textbf{Публичный} просит \textbf{Билетера}
                        снять бронь, и отвечает \textbf{Клиенту},
                        что тот проиграл
            \end{enumerate}
      \item Если ХЗ, что там с оплатой, то
            \begin{enumerate}
                  \item \textbf{Публичный} просит \textbf{Билетера}
                        снять бронь, и отвечает \textbf{Клиенту},
                        что тот проиграл
            \end{enumerate}
      \item ХЗ, как-то ждем, что-то делаем асинхронно,
            успокаиваем пользователя
\end{enumerate}

\subsection{Пользователь бронирует парковочное место}
Клиент бронирует парковочное место. В этом действии
принимает участие \textbf{Пользователем}. Для начала
нужно, чтобы \textbf{Пользователь} зарегистрирован на
сайте. \textbf{Пользователь} выбирает парковочное место.
\textbf{Пользователь} заполняет данные. \textbf{Пользователь}
одобряет транзакцию. Либо форма не отправляется/не доходит
до администратора.

\subsection{Пользователь регистрируется на сайте}
Клиент регистрируется на сайте (грузовом или пассажирском).
В этом действии участвует \textbf{Посетитель}.
\textbf{Пользователь} вводит почту и получает пароль,
\textbf{Пользователь} получает доступ к кабинету.
Либо Почта не дествительна аккаунт не регистрируется.

\subsection{Пользователь видит оповещение}
Клиент читает оповещение об изменении в работе аэропорта.
\textbf{Посетитель}, \textbf{Администратор}.
\textbf{Администратор} публикует оповещение.
\textbf{Пользователь} видит его на главной странице.
Администратор удаляет оповещение. Пользователю не
отображается оповещение. Либо Почта не дествительна
аккаунт не регистрируется.

\subsection{Пользователь читает новости}
Клиент читает оповещение об изменении в работе аэропорта.
В данном событии участвует \textbf{Пользователь}.
Он перешел на новостную страницу, переходит по ссылке на
новость и читает новость. Либо новость не найдена и
выводится ошибка.

\subsection{Пользователь пользуется картой сайта}
Клиент пользуется картой сайта. Актеры: Пользователь.
\textbf{Пользователь} перешел на страницу с картой сайта,
переходит по ссылке на страницу, попадает на нужную страницу.

\subsection{Пользователь печатает информационную страницу}
Пользователь запрашивает страницу на печать. Актеры:
\textbf{Посетитель}. Прежде всего \textbf{Пользователь}
должен находится на нужной странице. Пользователь нажимает
кнопку "печать", cайт создает копию страницы и печатает
страницу.

\subsection{Диаграммы}

\begin{figure}
      \includegraphics[width=16cm]{4-actions/use-case-payment.drawio.png}
      \centering
      \caption{Use-Case диаграмма: оплата билета}
\end{figure}

\begin{figure}
      \includegraphics[width=16cm]{4-actions/use-case.jpg}
      \centering
      \caption{Use-Case диаграмма: вся система}
\end{figure}


\section{Вывод}
Благодаря выполнению данной лабораторной работы,
мы смогли поработать над описанием процесса разработки
по методологии RUP, а также создали несколько use-case
диаграмм и описали прецеденты использования задананного
сервиса.

\end{document}
